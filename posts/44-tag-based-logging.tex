\documentclass{article}
\title{Tag-based logging}
\subtitle{#tags #logs}
\date{2025-08-19}
\modified{2025-08-19}
\keyword{programming}
\lobsters{https://lobste.rs/s/hx1dft/tag_based_logging}

\begin{document}

\section*

\epigraph{
  The problem we are facing is that the logs have devolved into noise
}{@Achilles, \href{https://stackoverflow.com/questions/2000323/noisy-logging-problem}{StackOverflow Question #2000323}}

In every large project I worked on, logging looked like a lost battle.
The raw logs contained so much data that developers gave up on reading them.
Instead, they dumped the logs into an indexing system (such as \href{https://grafana.com/docs/grafana/latest/panels-visualizations/visualizations/logs/}{Grafana})
and constructed elaborate queries to find the information they needed.
This setup created a split between local and production environments,
making local development painful.

Our logging model has barely evolved in decades---it's time for a change.
This article proposes an alternative approach to logging that could alleviate the pain.

\section{the-context-problem}{The context problem}

Log levels become progressively awkward as software gets more complex.
Firstly, the guidelines on how loud a log entry should be are contradictory, with different developers having widely different opinions on the matter.
\href{https://blog.codinghorror.com/the-problem-with-logging/}{Plenty}
\href{https://ntietz.com/blog/the-only-two-log-levels-you-need-are-info-and-error/}{of}
\href{https://www.reddit.com/r/golang/comments/yxiyt3/rethinking_logging_levels}{anecdotes}
and a broader analysis by Eduardo Mendes and Fabio Petrillo echo my experience:

\blockquote{
  We observed a lack of precision in the definitions of log severity levels.
  For example, there are severity levels in library definitions without distinguishing specific purposes
  or only characterized by adjectives and superlatives.
  This lack of precision can cause a misunderstanding of severity levels and hinder logging practices.
}{
  Eduardo Mendes, Fabio Petrillo, ``\href{https://arxiv.org/pdf/2109.01192}{Log severity levels matter: A multivocal mapping}''
}

Furthermore, log levels ignore context: information that seems irrelevant in one context becomes critical in another.
For example, blockchain nodes usually have two operating modes:
local testing mode (think \href{https://getfoundry.sh/anvil/overview/}{\code{anvil}})
and consensus mode (think \href{https://geth.ethereum.org/}{\code{geth}}).
When you run a node in development mode,
you want to see all events resulting from your actions:

\begin{code}
2025-07-26 12:00:00.123 |  warn | transaction rejected
    txid:   0x03b3ea998e6017dcf998261346e2a1412dd4c1883bc02432585af565ab29d7a3
    from:   0x03E8482c2BbF8247361b1D6000f138d9f9CA9CD7
      to:   0x999999cf1046e68e36E1aA2E0E07105eDDD1f08E
    value:  23000000
    reason: insufficient funds (balance: 1000000)

2025-07-26 12:00:01.235 |  info | transaction accepted
    txid: 0xc09a6e6a271a70c1853112929eb662e864008bb8a1fc9a4145c2ca817f28e9a1
    from: 0x03E8482c2BbF8247361b1D6000f138d9f9CA9CD7
      to: 0xc0ffee254729296a45a3885639AC7E10F9d54979
    args: 0x...

2025-07-26 12:00:02.002 |  warn | transaction failed
    txid:   0xc09a6e6a271a70c1853112929eb662e864008bb8a1fc9a4145c2ca817f28e9a1
    gas:    126_350
    reason: PreconditionFailed
\end{code}

Enabling such detailed logs in production would overwhelm the server.
A warning for a transaction submitter is a trace for a node operator.

Many more contexts exist in practice.
For example, as an owner of a specific system component,
I may want to inspect its performance-related traces.
Even advanced logging systems that can tune levels per component can't efficiently select only performance-related messages.
Enabling all component logs is ineffective since it would distort the measurements.
Ideally, the logging system should print only the messages I care about.

\section{tag-based-logging}{Tag-based logging}

The solution is simple: instead of grading events on a linear scale,
annotate them with arbitrary \emph{tags} like hashtags on social media.

\begin{code}
event!(#db, #perf, duration => duration, query = "fetch_transaction");
\cdots
event!(#user, txid => txid, ..., "transaction rejected");
\cdots
event!(#system, error => db_error, "database is corrupted");
\cdots
event!(#rpc, #perf, duration => duration, endpoint => "sendTransaction");
\end{code}

To select the events appropriate for our context, we construct a \emph{query},
a logical formula matching the event tags.

\begin{tabular}{ll}
Events we want to see & The query \\
Only events triggered by user actions & \code{#user} \\
All performance-related events & \code{#perf} \\
Database performance events & \code{#perf and #db} \\
Database or RPC-related events & \code{#db or #rpc} \\
Non-database performance events & \code{#perf and not #db} \\
\end{tabular}

This approach is strictly more powerful than the traditional logging scheme,
since log levels are a special case of tagging.
For example, the query for messages at the \code{info} level and higher would be \code{#info or #warning or #error}.

It is tempting to go further and support full \href{https://grafana.com/docs/loki/latest/query/log_queries/}{LogQL} in queries
and allow matching on arbitrary field values.
However, restricting the query language to statically-defined tags enables optimizations: The logging library can check queries without fully evaluating log records, making tag queries nearly as efficient as level comparisons\sidenote{sn-tracing-meta-optimization}{
  Rust’s `tracing` library uses a similar optimization:
  It relies on static event metadata to filter out uninteresting events,
  see \href{https://docs.rs/tracing/0.1.41/tracing/trait.Subscriber.html#tymethod.enabled}{\code{Subscriber::enabled}}.
}.
LogQL is essential for investigating existing logs,
but too flexible for efficiently selecting \emph{what} to log.

Tag-based logging has a theoretical foundation---\href{https://en.wikipedia.org/wiki/Formal_concept_analysis}{formal concept analysis}.
Tags correspond to \emph{attributes},
queries---to \emph{concepts},
and applying a query---to finding \emph{objects} modelling the concept.
Traditional log levels form a very boring concept lattice.

\section{tracing}{Querying logs with tracing}

Fortunately, we don't need to migrate entire infrastructures to new logging libraries:
Rust's excellent \href{https://crates.io/crates/tracing}{\code{tracing}} package is powerful enough to support a limited version of tags.

One piece of data we can use as a tag is the \code{target} attribute
that \code{tracing} attaches to every event.
It defaults to the current package name,
but the programmer can override it to be almost any string\sidenote{sn-target}{
  I believe that overriding the default target is a must if you're serious about your logs.
  Logs are a crucial application interface, and tying the target to your code structure can invalidate someone's log configurations if you decide to move the code around, which happened to me a few times.
}.

\begin{code}[rust]
tracing::debug!(target: "db", query = "fetch_transaction");
\end{code}


The environment variable controlling the logging (\code{RUST_LOG} by default) can serve as a primary mechanism for querying events.
\href{https://docs.rs/env_logger/latest/env_logger/#enabling-logging}{Traditionally},
Rust logging frameworks supported only the \code{[target][=][level][,...]} directives in the variable values,
limiting our query language to logical disjunctions with at most one tag per event
(for example, \code{#db or #rpc} becomes \code{db=debug,rpc=debug}).

Luckily, \code{tracing}'s \href{https://docs.rs/tracing-subscriber/0.3.19/tracing_subscriber/filter/struct.EnvFilter.html}{\code{EnvFilter}} supports a richer set of directives that can also select field names:
\code{target[span{field=value}]=level}.
This scheme suggests using event fields as tags:

\begin{code}
tracing::debug!(db=1, perf=1, ?duration, query = "fetch_transaction");
\end{code}

To construct a tag filter using this scheme,
we convert logical disjunctions into separate directives (\code{#a or #b} becomes \code{[\{a\}]=debug,[\{b\}]=debug})
and conjunctions into composite filters (\code{#a and #b} becomes \code{[\{a\}\{b\}]=debug}).
This scheme doesn't support negation (in other words, we cannot query events lacking a specific field) and thus our query language isn't \href{https://en.wikipedia.org/wiki/Functional_completeness}{functionally complete},
but it is good enough for most use cases.

\begin{tabular}{ll}
Tag query & RUST_LOG value \\
\code{#user} & \code{[\{user\}]=debug} \\
\code{#db and #perf} & \code{[\{db\}\{perf\}]=debug} \\
\code{#db or #rpc} & \code{[\{db\}]=debug,[\{rpc\}]=debug} \\
\code{#perf} & \code{[\{perf\}]=debug} \\
\code{(#db or #rpc) and #perf} & \code{[\{db\}\{perf\}]=debug,[\{rpc\}\{perf\}]=debug} \\
\end{tabular}

\code{EnvFilter} optimizes directives that don't refer to field values,
since these names are part of the static callsite metadata.
I wrote a tiny program to benchmark event filters
and couldn't detect any measurable difference in performance between queries based on target names and field names. 

We can mix both schemes, treating the \code{target} attribute as the primary tag,
and other field-tags as secondary.
This mixed approach complicates the query mapping a bit
(for example, we might have to write \code{db[\{perf\}]=debug} instead of \code{[\{db\}\{perf\}]=debug})
and forces the person configuring the logs to understand the distinction between primary and secondary tags.
However, it also works better with third-party libraries that rely on traditional log levels.
Furthermore, querying events with a single tag looks more natural if the tag is encoded as the event target.
In the end, the exact mapping scheme likely makes no difference as long as you apply it consistently.

If using \code{EnvFilter} is not an option,
the hierarchical nature of target matching offers an alternative solution.
Directive \code{db=debug} matches any target that starts with \code{db}, such as \code{db::perf}.
So using up to two tags per event and ordering them consistently enables logical conjunctions in our query language.
This approach can't query secondary tags alone---%
we must query either the primary tag only or both primary and secondary
(for example, we can query \code{#db and #perf}, but not \code{#perf}).

\begin{code}[rust]
debug!(target: "user", txid, ..., "transaction rejected");
debug!(target: "rpc::perf", ?duration, endpoint = "sendTransaction");
debug!(target: "db::perf", ?duration, query = "fetch_transaction");
\end{code}

\begin{tabular}{ll}
Tag query & RUST_LOG value \\
\code{#user} & \code{user=debug} \\
\code{#db and #perf} & \code{db::perf=debug} \\
\code{#db or #rpc} & \code{db=debug,rpc=debug} \\
\code{#perf} & n/a \\
\code{(#db or #rpc) and #perf} & \code{db::perf=debug,rpc::perf=debug} \\
\end{tabular}

Finally, we still need to decide what to do with \code{tracing}'s log levels.
We can't ignore them since the library forces us to pick a level for each entry.
I suggest using three log levels:

\begin{itemize}
\item Use \code{info} for one-time information useful in any context,
  such as an \textsc{http} server reporting its listening port.
\item Use the \code{error} level to report unrecoverable errors.
\item Use the \code{debug} level and tags for all other messages.
\end{itemize}

\section{non-linear-future}{Toward a non-linear future}

Linear log levels aren't the only reason logs became unwieldy, but they're a major contributor.
The tag-based logging model offers an alternative that requires ingenuity to fit into existing frameworks,
but \code{tracing} is powerful enough to get most of its benefits.

I started applying principles of tag-based logging in a large system and already feel the difference:
Finding the information I need became much easier.
Unsurprisingly, most of the benefits come not from sophisticated queries,
but from basic hygiene that the model imposes,
such as consistent use of log targets across components.

\end{document}
