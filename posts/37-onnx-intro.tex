\documentclass{article}

\title{ONNX introduction}
\subtitle{A gentle introduction to the Open Neural Network eXchange format.}
\date{2025-02-17}
\modified{2025-02-17}

\begin{document}

\section*

The machine learning ecosystem is a zoo of competing frameworks.
Each framework offers its custom model representation format, sometimes more than one:
\begin{enumerate}
\item
\href{https://pytorch.org/}{Pytorch} represents models using Python code that you can download from the \href{https://huggingface.co/models}{Hugging Face model repository}.
This format is flexible, but it is a disaster from the security point of view because it allows for arbitrary code execution.
\item
\href{https://www.tensorflow.org/}{TensorFlow} supports multiple formats, including \href{https://www.tensorflow.org/guide/saved_model}{SavedModel}.
The latest version is the \href{https://www.tensorflow.org/guide/keras/serialization_and_saving}{Keras} model format, which is a \textsc{zip}-file with the model components and metadata.
TensorFlow formats are underspecified, implementation-defined, and \href{https://github.com/tensorflow/tensorflow/blob/master/SECURITY.md}{generally insecure}.
\end{enumerate}

\textsc{onnx} is an open format for representing machine learning models;
it aims to unify the ecosystem.
It doesn't support advanced use cases, such as model training checkpoints,
but it is simple, secure (no arbitrary code execution), and easy to build on.

This article is an introduction to the \textsc{onnx} file format
I wish I had when embarked on my \textsc{ml} model transformation journey at \href{https://www.gensyn.ai/}{Gensyn}.

\section{anatomy}{The anatomy of an ONNX model}

The primary way to create an \textsc{onnx} file is to export it from a more popular model format,
such as Pytorch (using the \href{https://pytorch.org/docs/stable/onnx.html}{\code{torch.onnx}} module)
or TensorFlow (using third-party packages, such as \href{https://github.com/onnx/tensorflow-onnx}{\code{tf2onnx}}).

Conceptually, an \textsc{onnx} file describes a \emph{model}
containing a computation \emph{graph}.
The model acts as a program;
it establishes the context through imports and helper function definitions.
The primary graph is the \code{main} function;
it defines a \href{https://en.wikipedia.org/wiki/Pure_function}{pure function} that maps model inputs to outputs.

This description is abstract, so let's get practical and inspect the \href{https://onnx.ai/onnx/repo-docs/Syntax.html}{textual representation} of a tiny perceptron model.

\begin{figure}
\marginnote{mn-tiny-perceptron}{
  The textual representation of a tiny perceptron model.
}
\begin{code}[linenumbers]
<
    ir_version: 7,
    opset_import: ["" : 21]
>

G (float[N, 3] X) => (float[N, 2] Out)
<
    float[3, 4] W1 = {
        0.01, 0.02, 0.03, 0.04,
        0.05, 0.06, 0.07, 0.08,
        0.09, 0.10, 0.11, 0.12
    },
    float[4, 2] W2 = {
        0.11, 0.12,
        0.13, 0.14,
        0.15, 0.16,
        0.17, 0.18
    },
    float[4] B1 = { 0.001, 0.002, 0.003, 0.004 },
    float[2] B2 = { 0.01, 0.02 }
>
{
    Y1 = \href{https://onnx.ai/onnx/operators/onnx__Gemm.html}{Gemm}(X, W1, B1)
    Y2 = \href{https://onnx.ai/onnx/operators/onnx__Relu.html}{Relu}(Y1)
    Z = \href{https://onnx.ai/onnx/operators/onnx__Gemm.html}{Gemm}(Y2, W2, B2)
    Out = \href{https://onnx.ai/onnx/operators/onnx__Sigmoid.html}{Sigmoid}(Z)
}
\end{code}
\end{figure}

\begin{itemize}
\item
    Lines 1--4 specify \href{https://onnx.ai/onnx/repo-docs/IR.html#models}{model attributes}:
    the \textsc{onnx} format version and the versions of \href{https://onnx.ai/onnx/repo-docs/IR.html#operator-sets}{operator sets} (\emph{opsets}) this model uses.
    Refer to the \nameref{custom-operators} section for an explanation of the opset concept.
\item
    Line 6 defines the primary model graph named \code{G}.
    The graph takes one input named \code{X} (an \math{N \times  3} matrix of floats, where \math{N} is deduced from the input shape)
    and produces one output named \code{Out} (an \math{N \times  2} matrix of floats).
\item
    Lines 7--21 define the graph \emph{initializers} corresponding to the model weights.
    When you train an \textsc{onnx} model, you optimize the initializer values.
\item
    Lines 22--27 define the graph body, where each line (except for curly brackets) is an operator application.
    Each operator is a pure function mapping zero or more inputs to one or more outputs.
\end{itemize}

An \textsc{onnx} file is a model encoded as a \href{https://protobuf.dev/}{Protocol Buffers} message.
It's insightful to inspect this low-level representation of our example model.

\begin{figure}
\marginnote{mn-perceptron-raw}{
  A Protocol Buffers representation of a tiny \textsc{onnx} model.
  Ellipses in initializers do not belong to the message; they indicate omitted data entries.
}
\begin{code}
ir_version: 7
opset_import {
  domain: ""
  version: 21
}
graph {
  name: "G"
  input {
    name: "X"
    type {
      tensor_type {
        elem_type: 1
        shape {
          dim { dim_param: "N" }
          dim { dim_value: 3 }
        }
      }
    }
  }
  output {
    name: "Out"
    type {
      tensor_type {
        elem_type: 1
        shape {
          dim { dim_param: "N" }
          dim { dim_value: 2 }
        }
      }
    }
  }
  initializer {
    name: "W1"
    dims: 3
    dims: 4
    data_type: 1
    float_data: 0.01
    \ldots
  }
  initializer {
    name: "W2"
    dims: 4
    dims: 2
    data_type: 1
    float_data: 0.11
    \ldots
  }
  initializer {
    name: "B1"
    dims: 4
    data_type: 1
    float_data: 0.001
    \ldots
  }
  initializer {
    name: "B2"
    dims: 2
    data_type: 1
    float_data: 0.01
    \ldots
  }
  node {
    input: "X"
    input: "W1"
    input: "B1"
    output: "Y1"
    op_type: "Gemm"
    domain: ""
  }
  node {
    input: "Y1"
    output: "Y2"
    op_type: "Relu"
    domain: ""
  }
  node {
    input: "Y2"
    input: "W2"
    input: "B2"
    output: "Z"
    op_type: "Gemm"
    domain: ""
  }
  node {
    input: "Z"
    output: "Out"
    op_type: "Sigmoid"
    domain: ""
  }
  value_info {
    name: "Y1"
    type {
      tensor_type {
        elem_type: 1
        shape {
          dim { dim_param: "N" }
          dim { dim_value: 4 }
        }
      }
    }
  }
  value_info {
    name: "Y2"
    type {
      tensor_type {
        elem_type: 1
        shape {
          dim { dim_param: "N" }
          dim { dim_value: 4 }
        }
      }
    }
  }
  value_info {
    name: "Z"
    type {
      tensor_type {
        elem_type: 1
        shape {
          dim { dim_param: "N" }
          dim { dim_value: 2 }
        }
      }
    }
  }
}
\end{code}
\end{figure}

In this low-level representation, the graph has the following components:
\begin{itemize}
\item A list of \emph{inputs} that the caller must provide to compute the outputs.
\item A list of \emph{outputs} that the graph computes.
\item A list of \emph{initializers} specifying the model weights.
\item A list of \emph{nodes} sorted \href{https://en.wikipedia.org/wiki/Topological_sorting}{topologically}: a node can refer only to the graph inputs, initializers, and values that preceding nodes produce.
    Nodes refer to their inputs and outputs by names.
    Nodes might also have names, but all the nodes in our examples are unnamed.
\item A list of \code{value\_info} entries providing types for intermediate values.
\end{itemize}

Graph inputs, outputs, and initializers must have explicit types (in the example, these are float tensors of various shapes);
internal values might not have the corresponding \code{value\_info} entries.

\section{external-data}{External data}

Since Protocol Buffers \href{https://protobuf.dev/programming-guides/proto-limits/#total}{restrict the file size to two gigabytes}, \textsc{onnx} alone cannot encode large models\sidenote{sn-llama-size}{
    For example, \href{https://huggingface.co/meta-llama/Llama-3.1-8B}{Llama 3.1 \textsc{8b}} needs at least 16 gigabytes to encode its model weights using the \href{https://en.wikipedia.org/wiki/Bfloat16_floating-point_format}{\textsc{bf16}} type.
}.
To address this problem, \textsc{onnx} allows storing any tensor, such as an initializer or an operator attribute, in external files.

An external data reference specifies the location of the tensor file relative to the model file and the offset and the length within the file\sidenote{sn-offset-length}{
  The \code{offset} and the \code{length} fields are optional.
  Missing offset means ``from the beginning of the file,'' missing length---``until the end of the file.''
}.
The tensor is assumed to be in the flat row-major, little-endian format.

Let's make our tiny perceptron model larger and move its weight into a separate file.

\begin{figure}
\marginnote{mn-perceptron-external-tensors}{
  A perceptron model that stores its weights in an external file.
  The \code{location} path is relative to the model file.
}
\begin{code}[linenumbers]
<
    ir_version: 7,
    opset_import: ["" : 21]
>

G (float[N, 64] X) => (float[N, 10] Out)
<
    float[64, 1024] W1 = [
        "location": "weights.bin", "offset": "0", "length": "262144"
    ],
    float[1024, 10] W2 = [
        "location": "weights.bin", "offset": "262144", "length": "40960"
    ],
    float[1024] B1 = [
        "location": "weights.bin", "offset": "303104", "length": "4096"
    ],
    float[10] B2 = [
        "location": "weights.bin", "offset": "307200", "length": "40"
    ]
>
{
    Y1 = \href{https://onnx.ai/onnx/operators/onnx__Gemm.html}{Gemm}(X, W1, B1)
    Y2 = \href{https://onnx.ai/onnx/operators/onnx__Relu.html}{Relu}(Y1)
    Z = \href{https://onnx.ai/onnx/operators/onnx__Gemm.html}{Gemm}(Y2, W2, B2)
    Out = \href{https://onnx.ai/onnx/operators/onnx__Sigmoid.html}{Sigmoid}(Z)
}
\end{code}
\end{figure}

The textual syntax for external tensors desugars into a protobuf message
with the \href{https://github.com/onnx/onnx/blob/0277a1f62550c0b9edc3e1016a50a42dc4c73cf1/onnx/onnx.proto3#L653}{\code{data_location}} field set to \code{EXTERNAL},
and repeated \href{https://github.com/onnx/onnx/blob/0277a1f62550c0b9edc3e1016a50a42dc4c73cf1/onnx/onnx.proto3#L642}{\code{external_data}} fields indicating the data location.

\begin{figure}
\marginnote{mn-initializer}{
  An example of an initializer message that refers to external data.
}
\begin{code}
initializer {
  name: "W2"
  dims: 1024
  dims: 10
  data_type: 1
  external_data {
    key: "location"
    value: "weights.bin"
  }
  external_data {
    key: "offset"
    value: "262144"
  }
  external_data {
    key: "length"
    value: "40960"
  }
  data_location: EXTERNAL
}
\end{code}
\end{figure}

The external data feature explains why most \textsc{onnx} tools accept a path to the model file:
they might need to access external tensors,
and tensor locations are always relative to the model path.

The \href{https://onnx.ai/onnx/api/external_data_helper.html}{\code{onnx.external_data_helper}} Python module provides helpful utilities for dealing with external data.

\section{custom-operators}{Custom operators}

\textsc{onnx} allows the model to define custom operators, also called \emph{functions}.
The specification calls operator namespaces \emph{domains}
and groups operators into \href{https://onnx.ai/onnx/repo-docs/Versioning.html#operator-sets}{operator sets} (\emph{opsets}).
An opset is a versioned snapshot of operators from the same domain.

We know enough terminology now to understand the \code{opset_import} line at the top of our \textsc{onnx} programs.
It pins the exact operator semantics within the model graph.

The textual syntax for custom operators is almost identical to that of a graph definition.
Custom operator definitions must appear after the primary model graph
and have an attribute section defining their domain and dependencies.

The following example demonstrates a model that defines a custom operator doubling its input.

\begin{figure}
\marginnote{mn-double-example}{
  An \textsc{onnx} model that doubles its input using a custom operator.
}
\begin{code}[linenumbers]
<
    ir_version: 7,
    opset_import: ["" : 21, "com.example" : 1]
>

G (float[N] X) => (float[N] Out)
{
    Out = com.example.Double(X)
}

<
    domain: "com.example",
    opset_import: ["": 21]
>
Double (float[N] X) => (float[N] Out) {
    Out = \href{https://onnx.ai/onnx/operators/onnx__Add.html}{Add}(X, X)
}
\end{code}
\end{figure}

\begin{itemize}
\item
Line 3 imports two operator sets: the standard set and the custom set we define later in the program.
\item
Lines 11--17 define a custom operator that doubles its input.
Note the difference between lines 1--4 that set the top-level model attributes
and lines 11--14 that set attributes for the function that follows.
The operator attributes specify its domain and the opsets required for the implementation.
\end{itemize}

Nodes might have \emph{attributes} (values or subgraphs) that modify their behavior.
Custom operators can also define attributes.

The following examples introduce a custom \code{Root} operator that computes the \href{https://en.wikipedia.org/wiki/Nth_root}{n-th root} of its argument.

\begin{figure}
\marginnote{mn-onnx-root-example}{
    An \textsc{onnx} program that defines a custom \code{Root} operator with a single attribute, the root index.
}
\begin{code}[linenumbers]
<
    ir_version: 7,
    opset_import: ["" : 21, "com.example" : 1]
>

G (float[N] X) => (float[N] Out) {
    Out = com.example.Root<nth = 2>(X)
}

<
    domain: "com.example",
    opset_import: ["": 21]
>
Root <nth: int = 2> (float[N] X) => (float[N] Out) {
    One = \href{https://onnx.ai/onnx/operators/onnx__Constant.html}{Constant}<value_float = 1.0>()
    Nth = \href{https://onnx.ai/onnx/operators/onnx__Constant.html}{Constant}<value_int = @nth>()
    NthFloat = \href{https://onnx.ai/onnx/operators/onnx__Cast.html}{Cast}<to = 1>(Nth)
    E = \href{https://onnx.ai/onnx/operators/onnx__Div.html}{Div}(One, NthFloat)
    Out = \href{https://onnx.ai/onnx/operators/onnx__Pow.html}{Pow}(X, E)
}
\end{code}
\end{figure}

\begin{itemize}
\item
  Line 7 invokes our custom operator and explicitly specifies the \code{nth} attribute value.
\item
  Line 14 defines the \code{Root} operator with an attribute section.
\item
  Line 16 converts an attribute value into a constant graph node.
  The following line casts the integer value into a floating-point number.
\end{itemize}

\section{subgraphs}{Subgraphs}

\textsc{onnx} supports branching and looping using \emph{nested graphs} as operator attributes.
For example, the \code{If} operator accepts a single formal input---the condition---%
and two required attributes specifying the computation
that must happen in ``then'' and ``else'' branches.
Nested graphs can reference values from the outer scope.

There is no better way to demonstrate control flow than to solve \href{https://en.wikipedia.org/wiki/Fizz_buzz}{FizzBuzz} in \textsc{onnx}.

\begin{code}[linenumbers]
<
    ir_version: 7,
    opset_import: ["" : 21]
>

G (int64 Limit) => (string[N] Out) {
    Zero = \href{https://onnx.ai/onnx/operators/onnx__Constant.html}{Constant}<value_int = 0>()
    One = \href{https://onnx.ai/onnx/operators/onnx__Constant.html}{Constant}<value_int = 1>()
    Three = \href{https://onnx.ai/onnx/operators/onnx__Constant.html}{Constant}<value_int = 3>()
    Five = \href{https://onnx.ai/onnx/operators/onnx__Constant.html}{Constant}<value_int = 5>()
    Fifteen = \href{https://onnx.ai/onnx/operators/onnx__Constant.html}{Constant}<value_int = 15>()
    Cond = \href{https://onnx.ai/onnx/operators/onnx__Cast.html}{Cast}<to = 9>(One)

    Out = \href{https://onnx.ai/onnx/operators/onnx__Loop.html}{Loop} (Limit, Cond) <body = Body (int64 I, bool C) => (bool OutC, string Item) {
        X = \href{https://onnx.ai/onnx/operators/onnx__Add.html}{Add}(I, One)
        OutC = \href{https://onnx.ai/onnx/operators/onnx__Identity.html}{Identity}(C)

        M15 = \href{https://onnx.ai/onnx/operators/onnx__Mod.html}{Mod}(X, Fifteen)
        Z15 = \href{https://onnx.ai/onnx/operators/onnx__Equal.html}{Equal}(M15, Zero)
        Item = \href{https://onnx.ai/onnx/operators/onnx__If.html}{If} (Z15) <
            then_branch = FizzBuzz () => (string R) {
                R = \href{https://onnx.ai/onnx/operators/onnx__Constant.html}{Constant}<value_string = "fizzbuzz">()
            },
            else_branch = Other () => (string R) {
                M3 = \href{https://onnx.ai/onnx/operators/onnx__Mod.html}{Mod}(X, Three)
                Z3 = \href{https://onnx.ai/onnx/operators/onnx__Equal.html}{Equal}(M3, Zero)
                R = \href{https://onnx.ai/onnx/operators/onnx__If.html}{If} (Z3) <
                    then_branch = Fizz () => (string R) {
                        R = \href{https://onnx.ai/onnx/operators/onnx__Constant.html}{Constant}<value_string = "fizz">()
                    },
                    else_branch = Other () => (string R) {
                        M5 = \href{https://onnx.ai/onnx/operators/onnx__Mod.html}{Mod}(X, Five)
                        Z5 = \href{https://onnx.ai/onnx/operators/onnx__Equal.html}{Equal}(M5, Zero)
                        R = \href{https://onnx.ai/onnx/operators/onnx__If.html}{If} (Z5) <
                            then_branch = Buzz () => (string R) {
                                R = \href{https://onnx.ai/onnx/operators/onnx__Constant.html}{Constant}<value_string = "buzz">()
                            },
                            else_branch = Other => (string R) {
                                R = \href{https://onnx.ai/onnx/operators/onnx__Cast.html}{Cast}<to = 8>(X)
                            }
                        >
                    }
                >
            }
        >
    }>
}
\end{code}

\begin{itemize}
\item
  The textual representation doesn't support raw literals as operator arguments,
  so explicitly declare all the constants we'll use in lines 7--12.
  Constant nodes have no inputs and one output, the value they wrap.
  The magic constant \code{9} on line 12 is the \href{https://github.com/onnx/onnx/blob/0277a1f62550c0b9edc3e1016a50a42dc4c73cf1/onnx/onnx.proto3#L515}{boolean type id}.
\item
  \textsc{onnx} graphs express pure computation, so all operators must produce a value to be useful.
  Thus, \textsc{onnx} control structures resemble functional programming primitives, such as \href{https://hackage.haskell.org/package/base-4.21.0.0/docs/Data-List.html#v:unfoldr}{\code{unfoldr}}.

  The \code{Loop} operator accepts multiple arguments:
  the maximum number of iterations,
  the exit condition (the loop won't start if that value is \code{false}),
  and a sequence of internal loop variables (we don't use any in this example).
  The \code{body} graph attribute specifies the variable transformation at each step.

  The body transforms current iteration number, stop condition, and internal variables
  into the next exit condition (for early termination), the next values of internal variables, and an output value.
  The \code{Loop} operator accumulates all the output values into an output tensor.
  It returns the final values of intermediate variables and the accumulated outputs.

\item
  Our loop body graph is a sequence of nested \href{https://onnx.ai/onnx/operators/onnx__If.html}{\code{If}} operator calls.
  An \code{If} operator accepts a boolean condition value and two graph attributes: \code{then_branch} and \code{else_branch}.
  Our conditions check whether the input divides by 15, 5, and 3 and falls back to converting the number to string on line 39
  (\href{https://github.com/onnx/onnx/blob/0277a1f62550c0b9edc3e1016a50a42dc4c73cf1/onnx/onnx.proto3#L514}{type id 8} corresponds to the string type).
  Note how nested graphs can freely access values from their lexical scope.
\end{itemize}

\section{running-onnx-programs}{Appendix: running ONNX programs}

You can use the following Python code snippet to parse and play with the textual format
(the \href{https://docs.astral.sh/uv/guides/scripts/#running-scripts}{uv tool} makes it easy: \code{uv run --no-project script.py}).

\begin{figure}
\marginnote{mn-onnx-text-python-snippet}{
  A snippet of Python code that parses \textsc{onnx} textual syntax and runs the model.
  The model source is in bold.
}
\begin{code}[python]
\emph{# /// script
# dependencies = \[
#   "onnx~=1.17",
#   "onnxruntime~=1.18",
# \]
# ///}
import tempfile

import numpy as np
import numpy.typing as npt
import onnx
import onnx.external_data_helper
import onnx.parser
import onnxruntime as ort


def parse_onnx(text: str) -> onnx.ModelProto:
    model = onnx.parser.parse_model(text)
    onnx.checker.check_model(model)
    return onnx.shape_inference.infer_shapes(model, check_type=True)


def run_onnx(
    model: onnx.ModelProto, inputs: dict[str, npt.NDArray], outputs: list[str]
) -> dict[str, npt.NDArray]:
    onnx.external_data_helper.load_external_data_for_model(model, ".")
    with tempfile.NamedTemporaryFile() as model_file:
        onnx.save_model(model, model_file.name)
        runtime = ort.InferenceSession(model_file.name)
        return runtime.run(outputs, inputs)


print(
    run_onnx(
        parse_onnx("""\b{
    <ir_version: 7, opset_import: \["": 21\]>
    Square (float\[N\] X) => (float\[N\] Out) \{
        Out = Mul(X, X)
    \}
    }"""),
        inputs={"X": np.array([1.0, 2.0, 3.0], dtype=np.float32)},
        outputs=["Out"],
    )
)
\end{code}
\end{figure}

\section{resources}{Resources}
\begin{itemize}
\item \href{https://onnx.ai/onnx/intro/index.html}{The official \textsc{onnx} introduction.}
\item \href{https://onnx.ai/onnx/repo-docs/IR.html}{The \textsc{onnx ir} specification.}
\item \href{https://onnx.ai/onnx/repo-docs/Syntax.html}{\textsc{onnx} textual syntax.}
\item \href{https://github.com/onnx/onnx/blob/9fb11e344a7721b2eed1f3e26bf9312f168a79a0/onnx/test/cpp/parser_test.cc}{Tests for the textual syntax parser.}
\item \href{https://netron.app/}{The Netron \textsc{onnx} graph visualizer.}
\end{itemize}

% deleted
%
% One way to construct graphs programmatically is to use the \href{https://onnx.ai/onnx/api/helper.html#onnx.helper.make_node}{\code{onnx.helper}} \textsc{api}.
%
\end{document}
