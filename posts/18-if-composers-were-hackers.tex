\documentclass{article}

\title{If composers were hackers}
\subtitle{What programming language would J.S. Bach use?}
\date{2023-04-01}
\modified{2025-10-26}

\keyword{programming}
\keyword{music}
\keyword{glasperlenspiel}

\begin{document}
\section{introduction}{Introduction}

Music is my first true love\sidenote{zero-love}{Computer games are my zeroth true love.}.
The first time I got my hands on an old untuned guitar, I didn't want to put it down.
I found a summer job to buy my first instrument and pay for a teacher, and since then, music has been a constant source of joy in my life.
Picking a degree in computer science over a music college was a tough choice for me.

I can't help but look at the world through the lens of my music obsession.
There are many deep similarities between music and computing,
but we won't explore them in this article;
I will instead explore a silly and subjective question:
``If my favorite composers decided to write some code, which language would they pick?''

\section{bach}{Johann Sebastian Bach}

\href{https://en.wikipedia.org/wiki/Johann_Sebastian_Bach}{Johann Sebastian Bach} is arguably the most talented and prolific musician in history.
He would love the \href{https://en.wikipedia.org/wiki/APL_(programming_language)}{\textsc{apl}} programming language:

\begin{itemize}
  \item
    Bach's music is \emph{terse} like a well-written \textsc{apl} program.
    He achieves immense expressive power with a few well-chosen constructs.
  \item
    In his lifetime, Bach was famous primarily for his improvisation skills.
    \textsc{apl} is perfect for live coding because of its interactivity and terseness.
  \item
    Motifs and keys in Bach's music bear deep symbolic and emotional meaning.
    For example, a twisted motif might mean a crucifixion
    and a sequence of twelve beats might depict a clock striking midnight.
    Bach would appreciate how \textsc{apl} assigns meaning to symbols.
  \item
    Bach was phenomenally prolific, cranking at least one cantata (about 20 minutes of music) per week for Sunday church services for over three years,
    writing over three hundred cantatas in total,
    and it wasn't even his primary duty.
    Only a caffeinated \textsc{apl} wizard can match this pace.
  \item
    The \href{https://en.wikipedia.org/wiki/Contents_of_the_Voyager_Golden_Record}{Voyager Golden Record} contains three compositions by Bach.
    If we had to prove to aliens we can compute,
    \textsc{apl} is \emph{the} language to etch on a golden disk.
\end{itemize}

\begin{figure}[grayscale-diagram]
  \marginnote{mn-repr-bach}{
    Representative Bach's music: \emph{Wachet auf, ruft uns die Stimme}, a deeply moving and masterfully harmonized chorale.
  }
  \includegraphics{/images/18-bach-snippet.png}
\end{figure}
\begin{figure}
  \marginnote{mn-apl-program}{
    An \textsc{apl} function computing \math{2\sup{n}} \href{https://en.wikipedia.org/wiki/Gray_code}{Gray codes} for the given number \math{n}.
    This code comes from \href{https://www.jsoftware.com/papers/50/}{A History of \textsc{apl} in 50 functions} by Roger K.W. Hui.
    \href{https://tryapl.org/?clear&q=%7B(0%E2%88%98%2C%20%E2%8D%AA%201%E2%88%98%2C%E2%88%98%E2%8A%96)%E2%8D%A3%E2%8D%B5%E2%8D%89%E2%8D%AA%E2%8D%AC%7D%C2%A8%202%203%204&run}{Try it yourself!}
  }
  \center{
    \begin{code}[apl]{(0∘, ⍪ 1∘,∘⊖)⍣⍵⍉⍪⍬}\end{code}
  }
\end{figure}

\subsection{bach-apl-resources}{Resources}

If you want to learn more about J.S. Bach:
\begin{itemize}
  \item
    Listen to the \href{https://www.thegreatcourses.com/courses/bach-and-the-high-baroque}{Bach and the High Baroque} course\sidenote{sn-audible-gc}{
      \href{https://audible.com}{Audible.com} offers subscribers all courses I mention in this article for free or at a meager price.
    } by professor \href{https://robertgreenbergmusic.com/}{Robert Greenberg}.
  \item
    Read \href{https://www.amazon.com/Johann-Sebastian-Bach-Musician-Paperback-ebook/dp/B002GKGBLE}{Johann Sebastian Bach: The Learned Musician} by Christoph Wolff.
\end{itemize}

If you want to learn more about \textsc{apl}:
\begin{itemize}
  \item
    Read \href{https://en.wikipedia.org/wiki/Kenneth_E._Iverson}{Ken Iverson's} \textsc{acm} Turing Award lecture \href{https://dl.acm.org/doi/10.1145/358896.358899}{Notation as a Tool of Thought}.
    The \href{https://www.jsoftware.com/papers/}{jsoftware website} hosts this and many other papers on \textsc{apl}.
  \item
    Read \href{http://www.dyalog.com/mastering-dyalog-apl.htm}{Mastering Dyalog \textsc{apl}} by Bernard Legrand and play with the \href{https://tutorial.dyalog.com/}{Dyalog \textsc{apl} Tutorial}.
  \item
    Listen to the \href{https://arraycast.com/}{ArrayCast} and \href{https://apl.show/}{APL.Show} podcasts.
  \item
    Watch the \href{https://youtu.be/DsZdfnlh_d0}{Depth-first search in \textsc{apl}} video for inspiration.
  \item
    Consider getting a physical copy of \href{https://www.amazon.com/APL-Interactive-Approach-Leonard-Gilman/dp/0471093041}{APL: An Interactive Approach} book by Leonard Gilman and Allen J. Rose.
    The content is pretty dated, but it's one of the most engaging books on programming I've read.
\end{itemize}

\section{mozart}{Wolfgang Amadeus Mozart}

\href{https://en.wikipedia.org/wiki/Wolfgang_Amadeus_Mozart}{Mozart}'s musical genius was so bright and enigmatic that mysteries and myths still surround his life.

Mozart pairs well with the \href{https://www.scheme.org/}{Scheme} programming language.
\begin{itemize}
  \item
    Mozart's music reflects the values of \href{https://en.wikipedia.org/wiki/Age_of_Enlightenment}{the age of enlightenment}:
    it's clear and beautifully constructed.
    Scheme is just as simple and elegant: An introduction to the language fits in a few pages.
  \item
    Mozart mastered all genres of his time.
    Scheme is a programmable programming language;
    it is versatile enough to be helpful in any domain.
    And just like opera had a special place in Mozart's heart,
    programming language research sparks Scheme hackers' joy.
\end{itemize}
\begin{figure}[grayscale-diagram]
  \marginnote{mn-repr-mozart}{
    Representative Mozart's music: \href{https://www.youtube.com/watch?v=0rnJu1rlm90}{Piano Sonata 5 in G-major}, featuring a joyful melody with a simple but elegant arrangement.
  }
  \includegraphics{/images/18-mozart-snippet.png}
\end{figure}
\begin{figure}
  \marginnote{mn-scheme}{
    Symbolic differentiation in Scheme (\href{https://mitp-content-server.mit.edu/books/content/sectbyfn/books_pres_0/6515/sicp.zip/full-text/book/book-Z-H-16.html#%25_sec_2.3.2}{example 2.3.2} in \href{https://en.wikipedia.org/wiki/Structure_and_Interpretation_of_Computer_Programs}{SICP}).
    It's just as transparent and easy on the eyes as Mozart's works.
  }
  \begin{code}[scheme]
(define (deriv exp var)
  (cond ((number? exp) 0)
        ((variable? exp)
         (if (same-variable? exp var) 1 0))
        ((sum? exp)
         (make-sum (deriv (addend exp) var)
                   (deriv (augend exp) var)))
        ((product? exp)
         (make-sum
           (make-product (multiplier exp)
                         (deriv (multiplicand exp) var))
           (make-product (deriv (multiplier exp) var)
                         (multiplicand exp))))
        (else
         (error "unknown expression type -- DERIV" exp))))
\end{code}
\end{figure}
\subsection{mozart-scheme-resources}{Resources}

If you want to learn more about Mozart:
\begin{itemize}
    \item
    Listen to the \href{https://www.thegreatcourses.com/courses/great-masters-mozart-his-life-and-music}{Great Masters: Mozart---His Life and Music} course by professor \href{https://robertgreenbergmusic.com/}{Robert Greenberg}.
    \item
    Read \href{https://www.amazon.com/Mozart-Life-Maynard-Solomon/dp/0060883448}{Mozart: A Life} by Maynard Solomon.
\end{itemize}

If you want to learn more about Scheme:
\begin{itemize}
    \item
    Read \href{https://mitpress.mit.edu/9780262510875/structure-and-interpretation-of-computer-programs/}{The Structure and Interpretation of Computer Programs} by Harold Abelson and Gerald Jay Sussman.
    It's worth reading even if you don't care about Scheme.
    \item
    Read \href{https://www.amazon.com/Little-Schemer-Daniel-P-Friedman/dp/0262560992/}{The Little Schemer} and related books: \href{https://www.amazon.com/Reasoned-Schemer-MIT-Press/dp/0262535513/}{The Reasoned Schemer}, \href{https://www.amazon.com/Little-Typer-MIT-Press/dp/0262536439/}{The Little Typer}, \href{https://www.amazon.com/Little-Prover-MIT-Press/dp/0262527952/}{The Little Prover}, and \href{https://www.amazon.com/Little-Learner-Straight-Line-Learning/dp/026254637X/}{The Little Learner}.
\end{itemize}

\section{beethoven}{Ludwig van Beethoven}

\href{https://en.wikipedia.org/wiki/Ludwig_van_Beethoven}{Beethoven} is the most influential composer in history;
he single-handedly changed the direction of western music and the role of an artist in society.

One of the few languages worthy of Beethoven would be \href{https://www.haskell.org/}{Haskell}:
\begin{itemize}
  \item
  Beethoven constructed most of his music from tiny motifs and rhythms, many of which seem dull in isolation.
  Good Haskell code thrives on composition of small, often trivial functions like \href{https://hackage.haskell.org/package/base-4.18.0.0/docs/Prelude.html#v:id}{\code{id}} and \href{https://hackage.haskell.org/package/base-4.18.0.0/docs/Data-Function.html#v:fix}{\code{fix}}.
  \item
  Beethoven was constantly evolving his style, pushing the boundary of the art.
  He reinvented himself twice during his career, turning his suffering into fuel for breakthroughs.
  Similarly, the Haskell ecosystem constantly evolves, turning pain points into novel ways to write and think about software.
  For example,
  issues with lazy input/output led to the invention of \href{https://wiki.haskell.org/Iteratee_I/O}{iteratees},
  and the inconvenience of nested record updates led to \href{https://www.youtube.com/watch?v=k-QwBL9Dia0}{lenses} and \href{https://www.cis.upenn.edu/~bcpierce/papers/icmt-2009-slides.pdf}{bidirectional programming}.
  \item
  Beethoven turned my musical world upside down:
  I caught a piano bug after hearing \href{https://en.wikipedia.org/wiki/Piano_Sonata_No._17_(Beethoven)}{The Tempest} sonata finale.
  Haskell had a matching influence on my professional life,
  forever changing how I think about computing.
\end{itemize}

\begin{figure}[grayscale-diagram]
  \marginnote{mn-repr-beethoven}{
    Representative Beethoven's music: \href{https://www.youtube.com/watch?v=SrcOcKYQX3c&t=7s}{Sonata op. 13 in C minor}, also known as the Pathétique Sonata.
    Note the operatic drama, the masterful use of piano's sonority, and Beethoven's way of building music landscapes from tiny memorable motifs and rhythms.
  }
  \includegraphics{/images/18-beethoven-snippet.png}
\end{figure}

\begin{figure}
  \marginnote{mn-haskell-example}{
    An idiomatic implementation of the \href{https://en.wikipedia.org/wiki/Knuth%E2%80%93Morris%E2%80%93Pratt_algorithm}{Knuth-Morris-Pratt} algorithm in Haskell.
    The code comes from the \href{https://www.twanvl.nl/blog/haskell/Knuth-Morris-Pratt-in-Haskell}{Knuth-Morris-Pratt in Haskell} article by Twan van Laarhoven.
  }
  \begin{code}[haskell]
data KMP a = KMP { done :: Bool, next :: (a -> KMP a) }

makeTable :: Eq a => [a] -> KMP a
makeTable xs = table
   where table = makeTable' xs (const table)

makeTable' []     failure = KMP True failure
makeTable' (x:xs) failure = KMP False test
   where  test  c = if c == x then success else failure c
          success = makeTable' xs (next (failure x))

isSublistOf :: Eq a => [a] -> [a] -> Bool
isSublistOf as bs = match (makeTable as) bs
   where  match table []     = done table
          match table (b:bs) = done table || match (next table b) bs
\end{code}
\end{figure}

\subsection{beethoven-haskell-resources}{Resources}

If you want to learn more about Beethoven:
\begin{itemize}
  \item
  Listen to the \href{https://www.thegreatcourses.com/courses/great-masters-beethoven-his-life-and-music}{Great Masters: Beethoven---His Life and Music} course by professor \href{https://robertgreenbergmusic.com/}{Robert Greenberg}.
  \item
  Read \href{https://www.amazon.com/Beethoven-Revised-Maynard-Solomon/dp/0825672686}{Beethoven} by Maynard Solomon.
  \item
  Consider watching the \href{https://www.imdb.com/video/vi2964193561/}{Immortal Beloved} movie.
  It's inspiring despite its many factual mistakes.
\end{itemize}

If you want to learn more about Haskell:
\begin{itemize}
  \item
  Read \href{http://www.learnyouahaskell.com/}{Learn You a Haskell for Great Good!} by Miran Lipovača,
  an engaging illustrated introduction to Haskell.
  \item
  Read \href{https://www.euterpea.com/haskell-school-of-music/}{The Haskell School of Music} by Paul Hudak and Donya Quick;
  it demonstrates the best sides of Haskell by building a library for music generation.
  \item
  Check out the \href{https://www.haskell.org/documentation/}{Haskell documentation} page for more great pointers.
\end{itemize}

\section{scriabin}{Alexander Nikolayevich Scriabin}

\href{https://en.wikipedia.org/wiki/Alexander_Scriabin}{Scriabin} is a relatively obscure Russian composer
who wrote deeply expressive, emotional, metaphysical music.
He also had the most stylish mustache among all composers.
He would love \href{https://scala-lang.org/}{Scala}:

\begin{itemize}
  \item
  Scriabin dreamed of fusing arts into a synthetic form he called \href{https://en.wikipedia.org/wiki/Mysterium_(Scriabin)}{Mysterium}.
  Similarly, Scala aims to be extensible enough to embrace all programming styles and paradigms, including functional, object-oriented, and actor-based.
  \item
  Scriabin's music feels profound and unfathomable, like the night sky.
  Scala inspires the same awe: I admire its structure and elegance, even as its complexity makes my head spin.
\end{itemize}

\begin{figure}[grayscale-diagram]
  \marginnote{mn-repr-scriabin}{
    Representative Scriabin's music: \href{https://www.youtube.com/watch?v=Uy8MTTrh-Z8&t=709s}{Prelude in C#-minor from op. 11}, one of my favorite short piano pieces.
  }
  \includegraphics{/images/18-scriabin-snippet.png}
\end{figure}
\begin{figure}
\marginnote{mn-scala-snippet}{
  The Scala way of saying that optional values are just an instance of a monoid in the category of endofunctors.
  This snippet comes from the \href{https://github.com/scalaz/scalaz#type-class-instance-definition}{\code{scalaz}} library documentation.
}
\begin{code}[scala]
implicit val \emph{option}: Traverse[Option] with MonadPlus[Option] =
  new Traverse[Option] with MonadPlus[Option] {
    def \emph{point}[A](a: => A) = Some(a)

    def \emph{bind}[A, B](fa: Option[A])(f: A => Option[B]): Option[B] = fa flatMap f

    override def \emph{map}[A, B](fa: Option[A])(f: A => B): Option[B] = fa map f

    def \emph{traverseImpl}[F[_], A, B](fa: Option[A])(f: A => F[B])(implicit F: Applicative[F]) =
      fa map (a => F.map(f(a))(Some(_): Option[B])) getOrElse F.point(None)

    def \emph{empty}[A]: Option[A] = None

    def \emph{plus}[A](a: Option[A], b: => Option[A]) = a orElse b

    def \emph{foldR}[A, B](fa: Option[A], z: B)(f: (A) => (=> B) => B): B = fa match {
      case Some(a) => f(a)(z)
      case None => z
    }
}
\end{code}
\end{figure}

\subsection{scriabin-cl-resources}{Resources}

If you want to learn more about Scriabin:
\begin{itemize}
  \item
  Read \href{https://www.amazon.com/Alexander-Scriabin-Companion-History-Performance/dp/1442232617/}{The Alexander Scriabin Companion: History, Performance, and Lore} by Lincoln Ballard and Matthew Bengtson.
  \item
  Read \href{https://www.amazon.com/Scriabin-Biography-Second-Revised-Dover/dp/0486288978/}{Scriabin, a Biography} by Faubion Bowers.
\end{itemize}

If you want to learn more about Scala:
\begin{itemize}
    \item
    Read \href{https://www.amazon.com/Programming-Scala-Fifth-Odersky-dp-0997148004/dp/0997148004/}{Programming in Scala} by Martin Odersky et al., and \href{https://www.amazon.com/gp/product/1617290653/}{Functional Programming in Scala} by Rúnar Bjarnason and Paul Chiusano.
\end{itemize}

\section{closing}{Closing words}

There are many more great composers and excellent programming languages.
Matching these is left as an exercise for the reader\sidenote{sn-php-match}{Bonus points for finding a good match for \textsc{php}.}.

I like how hosts of the \href{https://kpknudson.com/my-favorite-theorem/}{My Favorite Theorem} podcast ask their guests to pair theorems with more tangible things and activities,
such as pizza and rock climbing.
Try it with the things or people you like, it's fun!
Enjoy your \href{https://www.youtube.com/watch?v=2G6dd7ikrXs}{favorite things}!

\end{document}
