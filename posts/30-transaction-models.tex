\documentclass{article}

\title{Transaction models are programming paradigms}
\subtitle{Functional programming vs OOP, now on a blockchain.}
\date{2024-08-16}
\modified{2024-08-16}
\keyword{blockchain}

\begin{document}
\section*

In late 2023, I had a lunch discussion with Thomas Locher,
a researcher at \textsc{dfinity} at the time,
with whom I worked on the \href{https://internetcomputer.org/ckbtc}{chain-key Bitcoin} project (ckBTC).
I told him that I find the \textsc{utxo} model quite elegant,
but Thomas argued that using the model for digital currency was the wrong decision
and that he couldn't see any benefits of using it.

I didn't have strong arguments to counter Thomas' statement, but
having built transaction managers for both Bitcoin and Ethereum,
I felt inclined toward Bitcoin's approach to transactions.

This article explores the differences between the account and \textsc{utxo} transaction models,
which are surprisingly similar to distinctions between programming styles.
Account-based chains model computation using the object-oriented style,
where each update touches the global ledger state,
while \textsc{utxo}-based chains employ functional style with \href{https://en.wikipedia.org/wiki/Substructural_type_system}{substructural typing}.
The benefits and downsides of each model mirror those of the corresponding programming styles.

\section{utxos-vs-accounts}{UTXOs vs accounts}

Many articles\sidenote{sn-utxo-articles}{
  See, for example, \href{https://docs.alchemy.com/docs/utxo-vs-account-models}{\textsc{utxo} vs. Account Models} from Alchemy.
} explain the differences between the transaction models from the end user's point of view.
In contrast, this section explores the engineering aspects of these models:
resistance to replay attacks, multi-way transactions\sidenote{sn-multi-way}{
  Since there is no official term for transactions with multiple inputs and outputs,
  I will call them \emph{multi-way transactions} in this article.
}, error recovery, and regulatory compliance.
The \textsc{utxo} model fairs well on all these dimensions.

\subsection{replay-attack-resistance}{Replay attack resistance}

A replay attack occurs when a malicious actor executes a transaction more than once.
Suppose Alice sends tokens to Eva in exchange for some goods.
If Eva could somehow resubmit Alice's transaction and receive double the amount,
she would have successfully mounted a replay attack.

Since account-based models contain generic instructions such as ``move X tokens from address A to address B,''
they require extra care to prevent replay attacks.
The Ethereum network requires each transaction to have a \emph{nonce}, a unique sequence number.
The network increments the sender's next expected nonce after every transaction.
This approach works, but it significantly complicates \href{#error-recovery}{error recovery}.

Some chains, such as \href{https://tron.network/}{Tron} and \href{https://internetcomputer.org/}{Internet Computer}, rely on time tracking to prevent replays,
allowing transactions to be valid only in a relatively short time window.
If the transaction validity range is outside the validity window, the system rejects the transaction.
When the system accepts a transaction, it adds the transaction to the deduplication pool and ignores all copies.
The system can safely remove a transaction from the pool once its validity range falls outside the active time window.

The \textsc{utxo} model solves the problem most elegantly.
Its transaction instructions are unambiguous: ``burn coins A and B and mint coin C,''
where A, B, and C are unique identifiers (the transaction hash and the output number).
Once a coin appears as input in a mined transaction,
all honest nodes reject further transactions referencing it as input.

\subsection{multi-way-transaction}{Multi-way transactions}

The \textsc{utxo} model allows transactions to have multiple inputs and outputs,
whereas the account model allows only point-to-point transfers.

The \textsc{dfinity} team employed multi-way transactions in the ckBTC token design.
Each \textsc{ic} user had a unique Bitcoin deposit address,
and the system could pool funds from several past deposits when someone wanted to withdraw their Bitcoin from the system.
This design increased the system transparency and made deposits cheap and convenient.

The next token the team worked on was \href{https://support.dfinity.org/hc/en-us/articles/20273018220180-What-are-ckETH-and-ckERC-20-tokens}{chain-key Ethereum} (ckETH).
Following the Bitcoin model,
the team wanted to have a unique deposit Ethereum address for each \textsc{ic} user,
but that design would require pooling funds from multiple accounts on withdrawals.
The team decided that submitting multiple transactions for a single withdrawal
introduces too much complexity and creates too many error conditions.
As a result, the team settled on a deposit flow
where users must identify themselves on each deposit by calling a helper smart contract.
This design led to increased deposit fees and made direct deposit from centralized exchanges impossible,
but it was the lesser of evils.

The main downside of multi-way transactions is the need for a robust \textsc{utxo} selection algorithm---%
a function that selects coins for a transaction with a given value.
Such an algorithm must solve a multi-objective optimization problem,
balancing the transaction size, the coin pool size, and the extra value locked in unconfirmed transactions\sidenote{sn-utxo-algorithms}{
  See \href{https://arxiv.org/abs/2311.01113}{A Survey on Coin Selection Algorithms in UTXO-based Blockchains} for a review of coin selection algorithms.
}.
Luckily, a simple greedy algorithm works quite well in practice,
so the benefits of a more complex transaction model outweigh the downsides in my book.

\subsection{error-recovery}{Error recovery}

Error recovery is the most challenging aspect of any distributed system.
In my experience, more than half of the code in a security-critical system can account for handling edge cases.

Nonce-based transaction models are most susceptible to failures requiring non-trivial intervention:
If a single transaction gets stuck, the system won't process any subsequent transactions.
The programmer has to keep track of the entire transaction queue and plan for recovery actions to unblock the queue.
This issue is not theoretical: I observed it first-hand while working with \textsc{evm} chains at Chainlink Labs,
where stuck transaction queues frequently cause oracle outages.

Time-scoped transactions allow for a simpler, localized recovery logic.
A single stuck transaction won't block other transactions from the same address,
and the system will heal itself with time
as blockchain nodes evict expired transactions from the mempool.

In the \textsc{utxo} model, failures are also local because they affect only coins participating in the failed transactions.
Further transactions can proceed normally unless a clever programmer decides to chain transactions opportunistically (i.e., use outputs from unconfirmed transactions in subsequent transactions) to reduce the system latency.

\subsection{regulatory-compliance}{Regulatory compliance}

A significant fraction of all tokens in circulation participated in questionable transactions (money laundering, for example).
Enforcement agencies, such as \href{https://en.wikipedia.org/wiki/Office_of_Foreign_Assets_Control}{\textsc{ofac}}, publish lists of sanctioned individuals and addresses.
All major exchanges integrate with chain analytics services (e.g., \href{https://www.chainalysis.com/}{Chainalysis}) that check whether incoming funds are ``tainted.''

In the account model, the account balance is a single number, and the tokens are genuinely fungible:
When an address receives a deposit, this deposit mixes with all the other tokens on the address.
A user cannot ``choose'' which tokens the system should use for the transfer.
On Ethereum, a malicious actor can taint your address by sending you tokens from a blocklisted account.

The \textsc{utxo} model packages tokens into distinct coins.
You can track each coin's history back to the block \href{https://en.bitcoin.it/wiki/Coinbase}{coinbase transaction} that minted its value.
Unique histories make coins non-interchangeable: some exchanges might refuse to accept ``tainted'' coins with a dubious history.
On Bitcoin, you can ignore coins you didn't expect to receive\sidenote{sn-tainted-coins}{
  By default, Bitcoin wallets will silently accept all coins and use them for payments
  unless you opt-in for \href{https://bitcoin.design/guide/how-it-works/coin-selection/#manual-coin-selection-aka-coin-control}{manual \textsc{utxo} selection}.
}, making tainting less effective.

The topic of tainted tokens is controversial and raises questions that will puzzle any rational mind\sidenote{sn-}{
  The \href{https://cryptoforensic.com/blog/tainted-bitcoin-isnt-what-you-think-it-is/}{Tainted Bitcoin Isn't What You Think It Is} article provides a good overview of the subject.
}.
We must accept regulations as a part of life; they are a product of fear and exist to make life complicated, not to make rational sense.
The \textsc{utxo} model deals with regulatory compliance slightly better because it allows ignoring unsolicited tainted coins.

\section{programming-models}{Programming models}

This section deals with extending the transaction models to support smart contracts.
The account model extends naturally to the message-passing programming style,
while the \textsc{utxo} model is similar to functional programming with resource types.

\subsection{objects-and-actors}{Objects and actors}

The Ethereum network was the first to extend blockchain technology to support arbitrary computations.
Ethereum virtual machine (\textsc{evm}) is close in spirit to the \href{https://en.wikipedia.org/wiki/Smalltalk}{Smalltalk} virtual machine.
Smart contracts are objects hiding their state and exchanging messages,
where the first message in the call chain comes from a user transaction.
There is no way to tell which parts of the blockchain state the transaction will touch without executing the transaction for real.

Continuing the analogy, the \href{https://internetcomputer.org/}{Internet Computer} blockchain (\textsc{ic}) relates to Ethereum
in the same way \href{https://www.erlang.org/}{Erlang} relates to Smalltalk.
Smart contracts on the \textsc{ic} are \href{https://en.wikipedia.org/wiki/Actor_model}{actors} that send and receive messages asynchronously.
To make this model viable, \textsc{ic} introduced \href{https://internetcomputer.org/capabilities/reverse-gas/}{reverse gas model}:
The burden of transaction fees shifted from users to smart contract operators.

Despite their differences, both systems share the fundamental programming model:
A smart contract is a stateful object reacting to incoming messages,
and a transaction is a request to invoke a message handler on that object with specified arguments.
In general, the exact effects of message processing are unpredictable.

\subsection{functions-and-resources}{Functions and resources}

Both \textsc{evm} and \textsc{ic} assume that a single transaction can arbitrarily modify the entire chain state.
The \textsc{utxo} model does not extend well in the same direction.
It packages state into independent values and requires that each transaction explicitly enumerates the resources it produces and consumes.

The most common generalization of the \textsc{utxo} model is to separate smart contract logic from the state,
turning the contract into a static rule for transforming transaction inputs into outputs.
That pattern corresponds to functional programming with \href{https://en.wikipedia.org/wiki/Substructural_type_system#Linear_type_systems}{linear types}.

Under the generalized \textsc{utxo} model,
a transaction is a functional expression that consumes resources (e.g., coins) created from previous transactions and produces new resources.
In Bitcoin, the resources can be only \textsc{btc} coins,
and the transition execution logic is hard-coded in the protocol.
A more general chain can allow \textsc{utxo}s to contain arbitrarily complex data structures and support arbitrary transition rules.

We can further allow immutable values that transactions can reference without consuming.
These \textsc{utxo} never become spent; they act as chemical catalysts transforming digital goods.
This feature comes in handy for storing transaction scripts on chain.

Unsurprisingly, blockchains based on the extended \textsc{utxo} model have deep roots in functional programming.
Two such chains, \href{https://cardano.org/}{Cardano} and \href{https://www.digitalasset.com/}{Digital Asset},
offer \href{https://www.haskell.org/}{Haskell}-inspired smart contract languages
(\href{https://developers.cardano.org/docs/smart-contracts/plutus/}{Plutus} and \href{https://github.com/digital-asset/daml}{\textsc{daml}}, respectively).


\section{conclusion}{Conclusion}

The \textsc{utxo} model was no mistake.
It works well with pure value transfers and offers a refreshing take on general-purpose programming on chain.

The extended \textsc{utxo} model offers benefits over the account model
that are similar to those that functional programming offers over object-oriented style:

\begin{itemize}
  \item
    \emph{Local reasoning.}
    Transaction effects are apparent from the transaction itself;
    there is no need to execute the transaction to understand what state it touches.
  \item
    \emph{Reproducibility.}
    The transaction submitter can check locally whether the transaction succeeds because its execution doesn't depend on the entire chain
    (under the assumption that transaction execution doesn't depend on the context, such as time or block number).
  \item
    \emph{Fewer unexpected interactions.}
    The blast radius of bugs in smart contracts is smaller because possible interactions within a transaction are limited.
  \item
    \emph{Better optimization opportunities.}
    The system can determine which transactions do not interfere based solely on their inputs and execute them in parallel.
\end{itemize}

The downsides of this model are the mirror image of their strengths.
The functional style is alien to most developers,
and it hinders complicated interactions and integrations that are dominant in mainstream blockchain programming.

Finally, the analogy between transaction and computation models explains why the \textsc{utxo} model felt more elegant to me:
My functional programming background primed me to like the model it resembles.
Functional programming with linear types is how I think about most problems.

\section{resources}{Resources}

\begin{itemize}
  \item \href{https://web.stanford.edu/class/ee374/lec_notes/lec4.pdf}{Transactions and the UTXO Model} lecture notes provide a general introduction to the \textsc{utxo} model.
  \item \href{https://files.zotero.net/eyJleHBpcmVzIjoxNzIxODU0MzcwLCJoYXNoIjoiYTVhYmY4NjdiY2E2YzdkNTNjODkwNWNmZDZhYmM5MjAiLCJjb250ZW50VHlwZSI6ImFwcGxpY2F0aW9uXC9wZGYiLCJjaGFyc2V0IjoiIiwiZmlsZW5hbWUiOiJDaGFrcmF2YXJ0eSBldCBhbC4gLSAyMDIwIC0gVGhlIEV4dGVuZGVkIFVUWE8gTW9kZWwucGRmIn0%3D/a3b27bc632152d8227dc27019859f91bb2a4ff8664fcf0a277cd90785570a2c2/Chakravarty%20et%20al.%20-%202020%20-%20The%20Extended%20UTXO%20Model.pdf}{The Extended UTXO Model}
  paper describes an extension of the \textsc{utxo} model introduced to support smart contracts on the Cardano network.
  \item The \href{https://www.digitalasset.com/hubfs/Canton/Canton%20Network%20-%20White%20Paper.pdf}{Canton Network} whitepaper describes how the network extends the \textsc{utxo} model.
\end{itemize}

\end{document}