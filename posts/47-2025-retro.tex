\documentclass{article}

\title{2025 retrospective}
\subtitle{Reading too much and writing too little.}
\date{2026-01-19}
\modified{2026-01-19}

\keyword{retrospective}

\begin{document}

\section*

2025 was an excellent year for me.
I \href{/posts/39-3000-days-of-duolingo}{quit Duolingo after 3000 days of practice},
completed my second marathon,
got promoted to a blue belt in \textsc{bjj},
read a few dozen books,
and understood more of category theory.
In 2026, I plan to do less and be more intentional.

\section{programming}{Programming}

Despite \textsc{llm}s being the hottest topic this year,
I didn't notice any seismic shifts in my daily work.
I listed my primary use cases for \textsc{llm}s
(making personalized tools, eliminating dependencies, and providing feedback)
in the \nameref{42-parasites-found} article.
Brian Cantrill's \href{https://rfd.shared.oxide.computer/rfd/0576}{Using LLMs at Oxide} also echoes my experience.

The worst downside of using \textsc{llm}s in a team for me
was an increase in the code review load and wasted effort.
A few times I had to review a multi-thousand-line code change
ridden with dubious code that obviously came from an \textsc{llm}.
Commenting felt so overwhelming and pointless
that I approved the code and addressed my concerns in follow-up changes.

On one hand, I don't want to slow down my colleagues
because of my potentially unreasonable quality bar.
On the other hand,
sloppy code slows \emph{everyone} down.
Rewriting committed code is a sure path to resentment,
but I don't see a good way out of this dilemma.
This problem existed before \textsc{llm}s,
but the \textsc{ai} hype made it more socially acceptable
to trade quality for speed.
Because allegedly, \textsc{llm} can fix the mess it created.

The most significant benefit of \textsc{llm}s for me
is psychological rather than technical.
Transitions are hard for me.
I struggle to start a task even
when I know exactly what I need to do and am motivated to do it,
and adding uncertainty exacerbates the problem.
So I ask Claude to do the task for me.
Its initial output is usually plain wrong or subpar,
but I'm already over the hump
and don't need much assistance\sidenote{sn-cunningham-law}{
  The \textsc{llm} version of \href{https://meta.wikimedia.org/wiki/Cunningham%27s_Law}{Cunningham's Law}: The best way to make me write good code is to show me bad code.}.

\section{books}{Books}

My most enjoyable read in 2025 was \href{https://www.goodreads.com/book/show/54493401-project-hail-mary}{Project Hail Mary} by Andy Weir.
I picked it up after watching the \href{https://youtu.be/5VYsnngkS_U}{upcoming movie trailer} that made my eyes wet.
I recommend the audio version read by Ray Porter, which \href{https://en.wikipedia.org/wiki/Audie_Award_for_Audiobook_of_the_Year#2020s}{won} the 2022 Audie Award for Audiobook of the Year.

My favorite technical book was \href{https://www.goodreads.com/book/show/63291820-rust-atomics-and-locks}{Rust Atomics and Locks} by Mara Bos.
Among the books on concurrency and memory models I've read,
that one is the clearest, the most practical, and the easiest to read.

I consumed a few dozen books on mental health and neurodivergence,
but these topics deserve a separate article.
Here are a few reading lists on other topics:

\subsection{books-pop-science}{Popular science}

\begin{enumerate}
\item
  \href{https://www.goodreads.com/book/show/5552.QED}{QED: The Strange Theory of Light and Matter} by Richard Feynman explains the interaction of light and matter in an engaging way.
  It is a paragon of popular science done right:
  it's easy enough for a dedicated layperson to follow,
  but deep enough to shatter your intuitive view of the world.
  For example, I learned that a \href{https://en.wikipedia.org/wiki/Diffraction_grating}{mirror} can defy intuitive \href{https://en.wikipedia.org/wiki/Reflection_(physics)#Laws_of_reflection}{reflection laws}:
  \blockquote{Isn’t it wonderful—you can take a piece of mirror where you didn't expect any reflection, scrape away part of it, and it reflects!}{
      Richard Feynman, QED, p. 47}
\item
  \href{https://www.goodreads.com/book/show/39001.Power_Sex_Suicide}{Power, Sex, Suicide: Mitochondria and the Meaning of Life} by Nick Lane
  gave me a deep appreciation of the little organelles that power our lives.
  I struggled to follow the details of chemical reactions
  and couldn't judge the soundness of Lane's arguments,
  but this book left me in awe.
  For example, I learned that the real difference between sexes is not the \href{https://en.wikipedia.org/wiki/Y_chromosome}{Y chromosome}\sidenote{sn}{Some females have Y chromosomes.},
  but the transfer of mitochondrial \textsc{dna} to the offspring. 
\item
  \href{https://www.goodreads.com/book/show/61475117-too-big-for-a-single-mind}{Too Big for a Single Mind: How the Greatest Generation of Physicists Uncovered the Quantum World} by Tobias Hürter
  tells the story of quantum mechanics that reads like a novel.
  It's not as illuminating as \textsc{qed},
  but it paints a fascinating picture of personalities standing behind the most significant discoveries in twentieth-century physics.
  I kinda hate Bohr now.
\end{enumerate}

\subsection{books-writing}{Writing}

\begin{enumerate}
\item
  \href{https://www.goodreads.com/book/show/186004.Stein_on_Writing}{Stein on Writing} by Sol Stein
  is packed with systematic advice on how to write prose that is hard to put down.
  It offers plenty of examples that make the advice's effects immediately apparent.
\item
  \href{https://www.goodreads.com/book/show/120549.Clear_and_Simple_As_the_Truth}{Clear and Simple As the Truth} by Francis-Noel Thomas and Mark Turner
  explains the concept of a writing style
  and analyzes the classical style in depth.
  I learned about it from Dan Luu's \href{https://danluu.com/writing-non-advice/}{Some thoughts on writing} blog post.
\item
  \href{https://www.goodreads.com/book/show/302021.Weinberg_on_Writing}{Weinberg on Writing}
  by Gerald M. Weinberg describes an approach to constructing prose from small ideas.
  It reminded me of the Zettelkasten method,
  but it's significantly less dogmatic.
\end{enumerate}

\subsection{books-development}{Personal development}
\begin{enumerate}
\item
  In \href{https://www.goodreads.com/book/show/857333.The_Art_of_Learning}{The Art of Learning},
  \href{https://www.joshwaitzkin.com/josh}{Josh Waitzkin} describes his journey from an international chess master
  to a tai chi pushing-hands world champion.
  It's an enjoyable read
  if you're interested in mastery and cross-pollination of interests.
\item
  On the surface, \href{https://www.goodreads.com/book/show/57468188-organizing-for-the-rest-of-us}{Organizing for the Rest of Us} by Dana K. White
  is a book about cleaning,
  but its guiding principles are universal.
  I found the \emph{container} concept especially deep and ubiquitous,
  as dealing with children's toys is surprisingly similar
  to deciding on project features
  or planning a busy day.
\item
  In \href{https://www.goodreads.com/book/show/32895535-why-buddhism-is-true}{Why Buddhism Is True}, Robert Wright argues
  that evolution shaped us to see reality distorted,
  and that the practices and core ideas of Buddhism can help us see more clearly
  and live calmer lives.
  I came to the same conclusions long ago,
  but it was still an enjoyable read.
  One idea from the book that stuck with me is that
  our feelings always carry judgments.
\end{enumerate}

\section{running}{Running}

In 2025, I logged 344 runs, 4295 kilometers in total.
That's 700 km more than in 2024,
but my performance traced a crooked trajectory.

\begin{figure}[grayscale-diagram]
\marginnote{mn-running-performance}{
   A self-assessment of my running performance in 2025.
}
\includegraphics{/images/47-running-performance.webp}
\end{figure}

The year started well as I was training for the Zürich Marathon.
The training routine felt challenging but manageable\sidenote{sn-hansons-method}{
  I used the program from the
  \href{https://www.goodreads.com/book/show/13592481-hansons-marathon-method}{Hansons Marathon Method} book.
},
and my form peaked on the marathon day.
My goal was to finish in 3:25:00;
the actual time was 3:25:35,
6:40 faster than the previous year.

The marathon day felt like a celebration.
I enjoyed most of the route,
and sustaining my target pace was easier than I expected.
Unlike last time, I didn't injure myself.

\begin{figure}
\marginnote{mn-marathon}{
  Me crossing the finish line.
  Reaching the destination after an exhausting run
  is the best feeling in the world.
}
\includegraphics{/images/47-marathon.webp}
\end{figure}

After the marathon,
I immediately started training
for a half-marathon.
After about a month,
my hard days became progressively more frustrating, as
I couldn't meet my relatively modest training goals.

On June 28th,
I had to cut my long Sunday run short and walk the four kilometers home.
That night,
I woke up with an intense pain that felt like a heart attack
and drove to an \textsc{er}.
My pain turned out to be \href{https://en.wikipedia.org/wiki/Pericarditis}{a complication} of a \href{https://en.wikipedia.org/wiki/Atypical_pneumonia}{walking pneumonia}.
I had to stay in a hospital for a few days for the first time in decades.

I recovered and slowly got back to running,
but I couldn't do any hard runs anymore.
Every time I attempted a threshold run or an interval session,
I got sick and had to slow down again.
My load consisted of \approx 80 km of slow-to-moderate running per week.

\begin{figure}[grayscale-diagram]
\marginnote{mn-running-weekly}{
    Total weekly running distance in 2025.
}
\includegraphics{/images/47-weekly-running.png}
\end{figure}

Weeks passed, and I felt no improvement.
I was short of breath and exhausted all the time.
Suspecting another bout of pneumonia,
I scheduled a doctor appointment,
and a blood test revealed such a low iron level
that the doctor offered me a 1000 mg infusion on the spot.
I took his offer and felt like a superman for two days,
but then I got sick again.

My body clearly wants me to prioritize rest and recovery.
I won't be training for a marathon in 2026.
I might participate in a shorter race around autumn if I feel better,
with no performance pressure.

\section{martial-arts}{Martial arts}

I continued practicing Brazilian Jiu-Jitsu (\textsc{bjj}) throughout the year,
logging around 250 hours.
It's still my favorite physical activity.

On June 14th, I participated in my first competition
(\href{https://smoothcomp.com/en/event/23831/bracket/1485667}{\textsc{sbjjnf} Summer Open 2025})
and won silver,
winning two rounds by submission \sidenote{sn-bjjnf-instagram}{
  My gym posted a \href{https://www.instagram.com/p/DLZNn-etsOU/}{short clip} on Instagram featuring the submissions.
}
and losing one by points (4-0).
The competition was a great experience
and greatly influenced my training.
Two weeks later, my coach recognized my progress by promoting me to a blue belt.

\begin{figure}
\includegraphics{/images/47-blue-belt.webp}
\end{figure}

I'll likely participate in another competition in 2026,
but I don't have any ambitions.
My only goal is to expose problems in my game.
I don't plan to travel anywhere to compete.

\textsc{bjj} also helped me meet good people in new places.
\textsc{bjj} practitioners seem to bond faster than ions in solution.
I trained in gyms in Geneva, Portugal, and Normandy during my business trips and vacations,
and I was always met with respect and friendliness.
Some gym owners even refused to charge money for my training sessions.

I also started learning the basics of \href{https://en.wikipedia.org/wiki/Judo}{judo}.
It's more straining and traumatic than \textsc{bjj},
but it's also more exquisite.
I might join a local judo club
to practice more systematically.

\section{writing}{Writing}

My most interesting and controversial article was
\nameref{38-static-types-perfectionism}.
Unfortunately, most of the feedback I got missed the point,
likely because people didn't read past the clickbait title and assumed
that I attacked static type checking.

I published far less than I wrote in 2025.
Some of my writing was too personal,
and a few gnarly topics refused to cooperate:
I couldn't construct an argument that was clear, convincing, and concise.
That means I have a lot of unfinished material,
and I'm looking forward to working on it in 2026.

\section{gaming}{Gaming}

These days, I play video games primarily with my daughters,
so almost all of them involve cats.
\href{https://store.steampowered.com/app/1177980/Little_Kitty_Big_City/}{Simulating} \href{https://store.steampowered.com/app/1332010/Stray/}{cats}, \href{https://store.steampowered.com/app/634160/Cattails__Become_a_Cat/}{hording cats}, \href{https://store.steampowered.com/app/2377970/Cats_Hidden_Around_the_World/}{searching cats}, \href{https://store.steampowered.com/app/1608230/Planet_of_Lana/}{solving} \href{https://store.steampowered.com/app/2781210/CAT__ONION/}{puzzles} \href{https://store.steampowered.com/app/1150950/Timelie/}{with cats}.

However, as a \href{https://www.hollowknight.com/}{Hollow Knight} fan,
I had to try \href{https://hollowknightsilksong.com/}{Hollow Knight: Silksong} released in September.
The game is brutally difficult,
but it's the best game I've ever played.
Hollow Knight is a masterpiece,
but Silksong is superior in every regard.

My kids loved it.
They role-play fights from the game and draw cute pictures of Hornet.
My older asks me to play it so that she can watch the game instead of the TV
(she can't play it herself,
but she loves using the controller to move Hornet around).
Sometimes she watches me struggling with a tough boss for an hour,
which is an opportunity to model my approach to solving problems:
be patient and learn the patterns until the solution comes almost effortlessly\sidenote{sn-sun-tzu}{
  Sun Tzu's advice is the key to Silksong mastery: ``First, learn to become invincible,
  then wait for your enemy's moment of vulnerability.''
  Learn to dodge, then sneak in attacks.
}.

My favorite aspect of the new game is the protagonist's personality.
In Hollow Knight, you play as a ghost, an empty vessel.
You don't know who you are and why you do what you do.
Your character is mute and bland.
In contrast, Silksong's Hornet has a strong voice.
She knows who she is,
where she comes from,
what she likes and wants,
and what she's good at.
\href{https://youtu.be/T_g3EGclng4?si=Wfr-LOn77-zs8g40&t=14}{She's clear about her boundaries.}
I felt like the Knight most of my life,
and I strive to be more like Hornet.

I needed 75 hours to beat the first two acts of the game.
It was a therapeutic journey,
and I won't rest until I beat the final third act.
And then I might consider playing a \href{https://store.steampowered.com/app/1809540/Nine_Sols/}{metroidvania with cats}.

\section{math}{Math}

One of my most proud achievements in 2025
was improving my grasp of \href{https://en.wikipedia.org/wiki/Category_theory}{Category Theory}.
I attribute it to a superior learning method:
I started a study group with my friend and ex-colleague Ulan.
We methodically worked through the book \href{https://www.goodreads.com/book/show/1810837.Basic_Category_Theory_for_Computer_Scientists}{Basic Category Theory for Computer Scientists} by Benjamin C. Pierce and solved all the exercises.
We met once a week and took turns explaining the solutions
and sharing analogies.


Social pressure helped me engage with the material
and persist through (a lot of) frustration.
I didn't become an expert in the subject,
but I went way further than I've ventured before.
The study group sessions will continue in 2026,
but we'll gradually shift our focus to another branch of math
(either linear algebra or group theory).

\section{music}{Music}

I practiced music significantly less than last year.
I learned two easy pieces by heart (the first movement of Mozart's Sonata No. 16 in C and Grieg's Op. 12 No. 5)
and practiced sight reading with no tangible improvement.

I tried both \href{https://www.sightreadingfactory.com/}{Sight Reading Factory} (\textsc{srf}) and \href{https://pianomarvel.com/en/}{Piano Marvel} to improve sight reading.
I found \textsc{srf}-generated music unbearably dull,
and it didn't automatically grade my performance.
Piano Marvel offered a much better experience,
and I made measurable progress\sidenote{sn-music-progress}{
  The app offers a \href{https://pianomarvel.com/en/feature/sasr}{SASR test feature} that throws progressively harder pieces at you and scores your performance.
} at first,
but its music library was somewhat limited,
and I didn't enjoy reading music from my phone's screen.

My attempts to improve sight-reading burnt me out.
I'm finally ready to admit to myself
that I'm exhausted and sleepy by the time I get to my music practice,
so I should use it for enjoyment and not expect any meaningful progress.

\section{most-memorable-day}{The most memorable day}

I had a few events in 2025 that I think fondly of,
such as the Zürich Marathon and my first \textsc{bjj} competition.
But somehow, the day that stands out the most is July 20th.
Like most memorable days, it was bittersweet.

That was a sunny Sunday.
My family was on vacation for a week,
so I was all by myself for the first time in a decade.
I thought I would savor the calm,
but it didn't feel special.
I went for a long run in the afternoon,
and it didn't go well.
I became exhausted after the first \textsc{10k}
and had to stop every few minutes to catch my breath.
That night, the chest pain returned,
and I had to stop exercising for a few days.

After I got home and rested,
I started working on the software that renders this blog.
I like math and occasionally use formulas in my articles.
I'm also averse to heavy dependencies and JavaScript.
Luckily, modern browsers natively support \href{https://w3c.github.io/mathml-core/}{MathML}.
I tried writing it by hand,
and it was unbearable.
Even $E = mc^2$ exploded into a pile of gibberish.
I needed a compiler that could convert TeX notation to MathML.

I'd wanted this feature for \href{https://github.com/roman-kashitsyn/mmapped.blog/issues/61}{years}
and browsed \href{https://tug.ctan.org/info/knuth-pdf/tex/tex.pdf}{TeX: The Program}, fishing for inspiration,
but never found the time to sit down and write the code.
On that Sunday, I had the time.
Four hours flew by before I got my first math expression rendered.
It took a few more sessions to polish the code and update the articles,
and I revised the parser a few times since then,
but all the groundwork happened on that day.

This Sunday reminded me how much I enjoy programming,
and how easy it is to sustain the flow state for hours
when the problem is meaningful and challenging.

\end{document}
